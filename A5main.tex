%%%%%%%%%%%%%%%%%%%%%%%%%%%%%%%%%%%%%%%%%%%%%%%%%%%%%%%%%%%%%%%
%
% Welcome to Overleaf --- just edit your LaTeX on the left,
% and we'll compile it for you on the right. If you open the
% 'Share' menu, you can invite other users to edit at the same
% time. See www.overleaf.com/learn for more info. Enjoy!
%
%%%%%%%%%%%%%%%%%%%%%%%%%%%%%%%%%%%%%%%%%%%%%%%%%%%%%%%%%%%%%%%


% Inbuilt themes in beamer
\documentclass{beamer}

% Theme choice:
\usetheme{CambridgeUS}

% Title page details: 
\title{ASSIGNMENT-5 \\ PROBABILITY AND RANDOM VARIABLES}  
\author{MARRI SATHVIKA \\ AI21BTECH11020}
\date{30th May,2022}

\begin{document}

% Title page frame
\begin{frame}
    \titlepage 
\end{frame}

\begin{frame}{QUESTION}
\textbf{Consider the following three events:}\\ (a) At least I six is obtained when six dice are rolled\\ (b) at least 2 sixes are obtained when 12 dice are rolled\\ (c) at least 3 sixes are obtained when 18 dice are rolled.\\Which of these events is more likely? 
\end{frame}

\begin{frame}{SOLUTION}
\textbf{Event(a):}\\
Probability of getting atleast 1 six obtained when six dice are rolled in event-a;\\
In rolling 6 dice : \\(i) Probability that no six will come out
\begin{align}
   Q_0 = (\frac{5}{6})^6 = 0.335 
\end{align}
(ii) Probability that 1 six will come out 
\begin{align}
   P_a = 1 - Q_0 = 1 - 0.335 = 0.665
\end{align}
\end{frame}

\begin{frame}{SOLUTION}
\textbf{Event(b): }\\
Probability of getting atleast 2 six obtained when 12 dice are rolled in event-b;\\
In rolling 12 dice : \\(i) Probability that no six will come out
\begin{align}
    Q_0 = (\frac{5}{6})^{12} = 0.112 
\end{align}
(ii) Probability that exactly 1 six will come out
\begin{align}
   Q_1=(\frac{1}{6})^1 (\frac{5}{6})^{11} \times 12 = 0.269 
\end{align}
(iii) Probability that at least 2 sixes will come out
\begin{align}
    P_b = 1-Q_0-Q_1
\end{align}
\begin{align}
    P_b = (1 - 0.112 - 0.269) = 0.619
\end{align}
\end{frame}

\begin{frame}{SOLUTION}
Probability of getting atleast 3 six obtained when 18 dice are rolled in event-c;\\
In rolling 18 dice : \\(i) Probability that no six will come out
\begin{align}
   Q_0 = (\frac{5}{6})^{18} = 0.038 
\end{align}
(ii) Probability that exactly 1 six will come out
\begin{align}
    Q_1 = (\frac{1}{6})^1 (\frac{5}{6})^{17} \times 18 = 0.135
\end{align}
(iii)Probability that exactly 2 sixes will come out
\begin{align}
    Q_2 = (\frac{1}{6})^2 (\frac{5}{6})^{16} \times _{18}C_2 = 0.230 
\end{align}
\end{frame}

\begin{frame}{SOLUTION}
Probability that at least 3 sixes will come out
\begin{align}
   P_c = 1 - Q_0 - Q_1 - Q_2
\end{align}
\begin{align}
   P_c = 1 - 0.38 - 0.135 - 0.230 = 0.597
\end{align}
 \begin{center}
    P_a > P_b > P_c\\
 \end{center}
 
 \begin{center}
    \boxed{\therefore $The most likely to happen is P_a$ } 
 \end{center}


\end{frame}
\end{document}